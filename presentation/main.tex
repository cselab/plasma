\documentclass[10pt,aspectratio=169]{beamer}
\beamertemplatenavigationsymbolsempty
\usepackage{graphicx}
\usepackage{amsmath}
\usetheme{default}

\begin{document}
\begin{frame}
  
\end{frame}

\begin{frame}
  \frametitle{What are we solving?}
  \begin{itemize}
    \item A system of 1D transport equations for tokamak plasmas.
    \item Evolves profiles of:
    \begin{itemize}
        \item Ion Temperature ($T_i$)
        \item Electron Temperature ($T_e$)
        \item Electron Density ($n_e$)
        \item Poloidal magnetic flux ($\psi$)
    \end{itemize}
    \item Self-consistently calculates:
    \begin{itemize}
        \item Plasma geometry
        \item Bootstrap current
        \item Conductivity
        \item Heating and particle sources
    \end{itemize}
  \end{itemize}
\end{frame}

\begin{frame}
  \frametitle{Numerical Scheme}
  \begin{itemize}
    \item \textbf{PDEs:} 1D (in $\rho$) coupled, nonlinear, second-order parabolic transport equations.
    \item \textbf{Spatial Discretization:} Finite Volume Method on a staggered grid.
    \begin{itemize}
        \item Diffusion: Central differencing.
        \item Convection: Upwind scheme with a Peclet number-dependent flux limiter.
    \end{itemize}
    \item \textbf{Boundary Conditions:}
    \begin{itemize}
        \item Core ($\rho=0$): Zero-gradient (Neumann).
        \item Edge ($\rho=1$): Fixed value (Dirichlet).
    \end{itemize}
    \item \textbf{Time Stepping:} Fully implicit backward Euler method.
    \begin{itemize}
        \item A predictor-corrector or Newton-Raphson method solves the nonlinear system.
    \end{itemize}
    \item \textbf{Coefficients \& Sources:} Self-consistently calculated from physics models (e.g., QLKNN, Sauter).
  \end{itemize}
\end{frame}

\begin{frame}[allowframebreaks]
  \bibliographystyle{abbrv}
  \bibliography{main}
  \nocite{*}
\end{frame}

\end{document}
